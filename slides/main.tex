\documentclass[utf8]{beamer}
\usepackage[utf8]{inputenc}
\usepackage[T2A]{fontenc}
\usepackage[english, russian]{babel}

\usepackage[style=verbose,backend=biber]{biblatex}
\addbibresource{references.bib}

% Настройка темы
\usetheme{Madrid}
\setbeamercolor{title}{fg=red!50!black, bg=blue!10}
\setbeamercolor{frametitle}{fg=white, bg=purple}
\setbeamercolor{block title}{fg=black, bg=yellow!30}

% Убираем дату полностью
\date{}

\usepackage{graphicx}

\setbeamertemplate{footline}{
  \hfill
  \usebeamercolor[fg]{page number in head/foot}%
  \usebeamerfont{page number in head/foot}%
  \insertframenumber{} / \inserttotalframenumber\hspace{1em}
  \vspace{2pt}
}

\setbeamertemplate{navigation symbols}{}

\begin{document}

\begin{frame}{Постановка задачи нейросетевого приёмника}
    \scriptsize
    \textbf{Дано:}
    \vspace{0.5em}
    \begin{itemize}
        \item SIMO-система с \( N_\text{t} = 1 \) передающими и \( N_\text{r} = 2 \) принимающими антеннами.
        \item Символы перед передачей: \( \mathbf{x} = \mathbf{W} \mathbf{s} \), где \( \mathbf{W} \) — прекодер, \( \mathbf{s} \) - исходная битовая последовательность, \( \mathbf{x} \) -  последовательность комплексных чисел.
        \item Модель канала: \( \mathbf{y} = \mathbf{H} \mathbf{x} + \mathbf{n} \), где \( \mathbf{H} \) — матрица канала, \( \mathbf{n} \) — белый шум.
        \item Имеются пилотные символы \( \mathbf{p} \).
    \end{itemize}
    \vspace{0.5em}
    \textbf{Найти:}
    \vspace{0.5em}
    \begin{itemize}
        \item Оценку переданных символов \( \hat{\mathbf{x}} \), используя нейросеть:
        \begin{equation}
        \hat{\mathbf{x}} = \mathcal{F}_\theta(\mathbf{y}, \mathbf{p}), \text{ где }\mathbf{y}\text{ - комплексные числа, прошедшие по каналу}
        \end{equation}
        \item Оптимальные параметры сети \( \theta \).
    \end{itemize}
    \vspace{0.5em}
    \textbf{Критерий качества:}
    \vspace{0.5em}
    \begin{itemize}
        \item Минимизация функции потерь:
        \begin{equation}
        \theta^* = \underset{\theta}{\mathrm{argmin}} \; \mathbb{E}_{(\mathbf{x},\mathbf{H},\mathbf{n})} 
        \left\{ \ell\left(\mathcal{F}_\theta(\mathbf{y}, \mathbf{p}),\, \mathbf{x} \right) \right\}, 
        \text{ где }\ell\text{ - кросс-энтропия}
        \end{equation}
        \item Контрольные метрики:
        \begin{equation}
        \text{BLER} = \frac{\text{Количество ошибочно переданных блоков}}{\text{Общее количество переданных блоков}}\text{ , }
        \end{equation}
        \begin{equation}
        \frac{E_b}{N_0} = \frac{\text{Энергия на один бит}}{\text{Спектральная плотность мощности шума}} \text{  [дБ]}
        \end{equation}
    \end{itemize}
\end{frame}

\begin{frame}{Модификация приёмника и сравнительный анализ}
    \scriptsize
    \begin{itemize}
        \item \textbf{Модификация нейросетевого приёмника: } В качестве модификации нейросетевого приёмника были применены результаты статьи \cite{woo2018cbam}
        \item \textbf{Сравнительный анализ: } Было проведено сравнение модификации нейросетевого приёмника со следующими моделями: идеальный приёмник (англ. Perfect CSI), приёмник на основе метода наименьших квадратов (англ. LS estimation), нейросетевой приёмник (англ. Neural Receiver) - на кластеризованном канале линии задержки (англ. CDL channel).
    \end{itemize}

    \centering
    \includegraphics[width=0.75\textwidth]{result_E (2).pdf}
    \scriptsize
    \vspace{0.5em}

\end{frame}



\end{document}