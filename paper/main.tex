\documentclass{article}
\usepackage{arxiv}

\usepackage[utf8]{inputenc}
\usepackage[english, russian]{babel}
\usepackage[T1]{fontenc}
\usepackage{url}
\usepackage{booktabs}
\usepackage{amsfonts}
\usepackage{nicefrac}
\usepackage{microtype}
\usepackage{lipsum}
\usepackage{graphicx}
\usepackage{natbib}
\usepackage{doi}

\usepackage[warn]{mathtext}
\usepackage[T2A]{fontenc}			% кодировка
\usepackage[utf8]{inputenc}			% кодировка исходного текста
\usepackage[english,russian]{babel}	% локализация и переносы
\usepackage{indentfirst}
\usepackage{csquotes}
% \usepackage[bibstyle=gost-numeric, sorting=none]{biblatex}
% \addbibresource{biblio.bib}

\linespread{1.3}

\usepackage[utf8]{inputenc}
\usepackage[T2A]{fontenc}
\usepackage[russian]{babel}
\usepackage{hyphenat} 

\usepackage{array}
\usepackage{multirow}
\usepackage{xcolor}
\usepackage{colortbl}

% page settings
\usepackage[
    left=1.5cm,
    right=1.5cm,
    top=1.5cm,
    bottom=1.5cm,
    bindingoffset=0cm
]{geometry}

\usepackage{graphicx, hyperref, xcolor}
\hypersetup{
    colorlinks=true,
    linkcolor=teal,
    filecolor=magenta, 
    urlcolor=blue,
    citecolor=olive,
    pdftitle={GD},
    % pdfpagemode=FullScreen,
    linktoc=all
    }

\usepackage{wrapfig,caption}

% figures
\usepackage{caption}
\usepackage{subcaption}
\usepackage{floatrow}
\floatsetup{heightadjust=object}

% math
\usepackage{amsmath,amsfonts,amssymb,amsthm,mathtools,esint,eucal}


\def\vec#1{\mathchoice{\mbox{\boldmath$\displaystyle#1$}}
{\mbox{\boldmath$\textstyle#1$}} {\mbox{\boldmath$\scriptstyle#1$}} {\mbox{\boldmath$\scriptscriptstyle#1$}}}

\title{Применение машинного обучения в 5G. Нейросетевой приёмник и его CBAM модификация.}

\author{Овсиенко О. П. \\
	МГУ им. М.В. Ломоносова\\
	факультет ВМК\\
        119991, Москва, Ленинские горы, д.1, стр. 52\\
	\texttt{st02220041@gse.cs.msu.ru} \\
        \And
        Бобров Е. А.\\
	МГУ им. М.В. Ломоносова\\
	факультет ВМК\\
        119991, Москва, Ленинские горы, д.1, стр. 52\\   
}
\date{}

\renewcommand{\shorttitle}{\textit{arXiv} Template}

\hypersetup{
pdftitle={My first paper Ovsienko},
pdfkeywords={neural receiver, CBAM, 5G},
}

\begin{document}
\maketitle

\begin{abstract}
В статье исследуется применении методов машинного и глубокого обучения для оптимизации физического уровня (англ. PHY Layer) в современных системах мобильной связи пятого поколения (англ. 5G). Рассмотрены традиционные методы обработки сигналов, такие как линейная эквализация (англ. LMMSE equalizer) и оценка канала методом наименьших квадратов (англ. LS channel estimator), а также нейросетевые подходы, включая архитектуру с остаточными блоками от NVIDIA. Центральным результатом работы стала разработка модификации нейросетевого приёмника с модулем внимания свёрточного блока (англ. Convolutional Block Attention Module, CBAM), которая продемонстрировала превосходство над классическими методами в условиях многолучевого распространения и помех. Практическая значимость исследования заключается в возможности интеграции предложенного метода в современные системы связи для повышения их эффективности и устойчивости к помехам.
\end{abstract}

\keywords{традиционные методы обработки сигналов\and нейросетевой приёмник}

\section{Введение}

Современные системы мобильной связи пятого поколения сталкиваются с необходимостью обеспечения исключительно высокой скорости передачи данных, сверхнизких задержек и надёжного соединения для миллионов устройств. Эти требования предъявляют новые вызовы к методам обработки сигналов на PHY Layer, особенно в условиях многолучевого распространения и интерференции. Традиционные алгоритмы, такие как LS channel estimator и LMMSE equalizer (\citet{badr2019lmmse}), зачастую не справляются с растущей сложностью каналов, что стимулирует поиск более совершенных и адаптивных подходов.

Классические подходы к обработке сигналов в системах связи с
несколькими входами
и несколькими выходами (англ. MIMO) в сочетании
с
технологией мультиплексирования с ортогональным разделением частот (англ. OFDM) (\citet{bagadi2010mimo}), такие как оценка канала на основе пилот-сигналов и последующий LMMSE equalizer, хорошо изучены и обеспечивают приемлемое качество в условиях умеренных помех. Однако их эффективность существенно снижается в сценариях с сильной интерференцией, быстрыми замираниями и в отсутствие точной информации о состоянии канала (англ. Channel State Information, CSI). LS channel estimator, будучи вычислительно простым, крайне чувствителен к шуму, в то время как более робастные методы, такие как LMMSE equalizer, требуют априорного знания статистики канала и шума, что зачастую недостижимо в реальных динамичных условиях.

В последние годы активно развиваются методы машинного и глубокого обучения, демонстрирующие значительный потенциал для оптимизации телекоммуникационных систем. В частности, нейросетевые приёмники, способные обучаться сквозному преобразованию принятого сигнала в битовые последовательности, показывают результаты, превосходящие классические методы. Одной из таких современных архитектур является нейросетевой приёмник от NVIDIA (\citep{sionna}), построенный на основе остаточных свёрточных блоков (англ. Residual Blocks), который позволяет эффективно компенсировать искажения сигнала без явного знания матрицы канала.

В данной работе предлагается модификация данного нейросетевого приёмника, направленная на дальнейшее повышение его эффективности. Модификация заключается во внедрении механизма внимания CBAM (\citet{woo2018cbam}) в структуру остаточных блоков. Этот механизм позволяет сети адаптивно выделять наиболее информативные признаки в частотно-временном и канальном пространствах, усиливая полезные компоненты сигнала и подавляя шумовые.

Экспериментальные исследования, проведённые на кластеризованных каналах линии задержки (англ. Clustered Delay Line Channel, CDL), подтвердили эффективность предложенного подхода (\citet{pessoa2020cdl}). Модифицированный приёмник продемонстрировал устойчивое улучшение показателя частоты ошибочных блоков (англ. Block Error Rate, BLER) по сравнению как с классическими методами (LS channel estimator, LMMSE equalizer), так и с базовой нейросетевой архитектурой, в особенности в условиях нелинейной видимости (англ. NLOS) и при низких отношениях сигнал/шум. Таким образом, вклад работы заключается в разработке и валидации усовершенствованной архитектуры приёмника, что открывает пути для создания более помехоустойчивых и спектрально-эффективных систем связи следующего поколения.

\section{Обзор литературы}

Современные системы связи пятого поколения требуют принципиально новых подходов к обработке сигналов на физическом уровне. Традиционные методы, такие как оценка канала и эквализация, хорошо изучены, но имеют существенные ограничения в условиях сложных сценариев распространения сигнала. \citet{badr2019lmmse} проводит всесторонний анализ LMMSE equalize оценки канала в контексте OFDM систем, подчеркивая её преимущества перед более простыми методами, такими как LS channel estimator, но также отмечая зависимость от точного знания статистических характеристик канала. Исследование \citet{bagadi2010mimo} демонстрирует применение пилот-сигналов для оценки канала в системах MIMO-OFDM, что является стандартным подходом в современных системах связи.

Архитектуры MIMO систем играют ключевую роль в повышении пропускной способности и надежности беспроводных каналов. \citet{shah2017performance} предоставляет сравнительный анализ различных конфигураций MIMO, включая системы с одним входом и одним выходом (англ. SISO), с одним входом и несколькими выходами (анг. SIMO), с несколькими входами и одним выходом (англ. MISO) и MIMO, показывая преимущества пространственного разнесения и мультиплексирования. Дальнейшее развитие MIMO технологий представлено в работе \citet{pessoa2020cdl}, где предлагается модель канала на основе CDL с двухполяризованными антеннами, специально адаптированная для сценариев 5G в сельской местности.

Теоретические основы достижимых скоростей передачи в системах связи продолжают развиваться. \citet{bjornson2015achievable} исследует предельные возможности Rician MIMO каналов с учетом аппаратных искажений, что особенно актуально для систем массового MIMO. В то же время \citet{caire2017achievable} фокусируется на методах вероятностного формирования сигналов (probabilistic shaping) для повышения спектральной эффективности.

Прорыв в области глубокого обучения открыл новые перспективы для оптимизации физического уровня. Архитектура глубоких остаточных свёрточных сетей, представленная в работе \citet{he2016deep}, стала фундаментом для многих последующих разработок, включая нейросетевые приемники. Механизмы внимания, такие как CBAM из исследования \citet{woo2018cbam}, демонстрируют эффективность в выделении информативных признаков в компьютерном зрении, что может быть успешно перенесено на задачи обработки сигналов.

Современные инструменты для исследований в области связи, такие как Sionna от NVIDIA \citep{sionna}, предоставляют мощную платформу для сквозного моделирования систем связи и интеграции методов машинного обучения. В частности, \citet{sionna2023neural} демонстрирует применение нейросетевых приемников для OFDM SIMO систем. Исследования \citet{levis2020deeprx} и \citet{levis2020end} развивают концепцию полностью сверточного глубокого приемника и сквозного обучения для OFDM систем, включая возможность беспилотной коммуникации.

Важным аспектом современных систем связи являются эффективные методы кодирования. Работы \citet{nguyen2019efficient} и \citet{nguyen2021low} посвящены разработке низкосложных декодеров QC-LDPC для 5G, что критически важно для практической реализации систем следующего поколения. Перспективным направлением является также совместное демодулирование сложных созвездий с использованием машинного обучения, как показано в исследовании \citet{gansekoele2025joint}.

Таким образом, современное состояние исследований демонстрирует четкий тренд к интеграции методов машинного обучения с классическими подходами обработки сигналов, что открывает новые возможности для создания более эффективных и устойчивых систем связи.

\section{Математическая постановка задачи}
Рассматривается система MIMO с \( N_\text{t} \) передающими антеннами и \( N_\text{r} \) принимающими антеннами. Пусть:
\begin{equation}
\mathbf{x} = \mathbf{W} \mathbf{s}
\end{equation}
где \(\mathbf{W} \in \mathbb{C}^{N_\text{t} \times N_\text{r}}\) – матрица прекодера, \(\mathbf{s} \in \mathbb{C}^{N_\text{r} \times 1}\) – вектор исходных сигналов ( бинарная последовательность ), а \(\mathbf{x} \in \mathbb{C}^{N_\text{t} \times 1}\) - передаваемый вектор сигналов после прекодера (каждый символ в пропорции распространён по всем антеннам). Этот вектор обычно берётся из некоторой M-QAM/QPSK сигнальной диаграммы, т.\,е. \( \mathbf{x} \) состоит из комплексных значений, отражающих точку созвездия.

Модель прохождения сигнала через канал можно записать в следующем виде:

\begin{equation}
\mathbf{y} = \mathbf{H} \mathbf{x} + \mathbf{n},
\end{equation}
где
\[
\mathbf{y} \in \mathbb{C}^{N_\text{r} \times 1}
\]
– вектор принятых сигналов на антенных элементах приёмника, \(\mathbf{H} \in \mathbb{C}^{N_\text{r} \times N_\text{t}}\) – матрица канала, описывающая затухание и фазовые сдвиги на пути «от каждой передающей антенны к каждой приёмной», а \(\mathbf{n} \in \mathbb{C}^{N_\text{r} \times 1}\) – вектор аддитивного белого гауссовского шума (AWGN) с дисперсией \(\sigma_n^2\).

В классической постановке задачи \textbf{MIMO-декодирования} (или «символьной детекции») требуется найти оценки \(\hat{\mathbf{x}}\) переданных символов, обладающие минимальной вероятностью ошибки. При стандартном подходе (например, методе MMSE – Minimum Mean Square Error) оценивается:
\begin{equation}
\hat{\mathbf{x}}_\mathrm{MMSE} = \underset{\mathbf{x}}{\mathrm{argmin}}( 
\Big\lVert \mathbf{y} - \mathbf{H}\mathbf{x} \Big\rVert^2 + \sigma_n^2 \lVert \mathbf{x} \rVert^2).
\end{equation}
Для высоких размерностей (большое число антенн \(N_\text{t}\) и \(N_\text{r}\)), а также для сложных созвездий (например, 64-QAM, 256-QAM) точный поиск \(\mathbf{x}\) может становиться вычислительно дорогостоящим. Методы глубокого обучения позволяют решать данную задачу итеративно или напрямую, обучаясь на большом наборе «принятый сигнал – истинный символ» без явной оптимизации по формуле минимума ошибки.

В рамках предлагаемой работы рассматривается ситуация, когда матрица канала \(\mathbf{H}\) частично неизвестна или может изменяться со временем, а число передатчиков и приёмников велико. Требуется реализовать «глубокую» нейронную сеть, которая сможет, во-первых, оценивать параметры \(\mathbf{H}\) по пилотным символам, а во-вторых, автоматически восстанавливать \(\mathbf{x}\) из \(\mathbf{y}\) с учётом сложности канала и помех.

\textbf{Основная цель} формализуется как задача:
\begin{equation}
\hat{\mathbf{x}} = \mathcal{F}_\theta(\mathbf{y}, \mathbf{p}),
\end{equation}
где \(\mathbf{p}\) – набор пилотных символов, а \(\mathcal{F}_\theta\) – функция, задаваемая параметрами нейронной сети \(\theta\). 

Требуется найти такие параметры \(\theta\), которые минимизируют среднюю вероятность ошибки детектирования:
\begin{equation}
\theta^* = \underset{\theta}{\mathrm{argmin}} \; \mathbb{E}_{(\mathbf{x},\mathbf{H},\mathbf{n})} 
\Big\{ \ell\big(\mathcal{F}_\theta(\mathbf{y}, \mathbf{p}),\, \mathbf{x}\big) \Big\},
\end{equation}
где \(\ell(\cdot)\) – функция потерь (кросс-энтропия).

Для оценки качества декодирования в моделях связи используются BLER и $\frac{E_b}{N_0}$: 

\begin{equation}
        \text{BLER} = \frac{\text{Количество ошибочно переданных блоков}}{\text{Общее количество переданных блоков}}\text{ , }
        \end{equation}
        \begin{equation}
        \frac{E_b}{N_0} = \frac{\text{Энергия на один бит}}{\text{Спектральная плотность мощности шума}} \text{  [дБ]}
\end{equation}

Ошибочный блок — это блок, в котором хотя бы один бит не совпадает с исходным. BLER помогает отслеживать эффективность модели, показывая, насколько часто возникают ошибки в передаваемых данных. Чем ниже BLER, тем лучше модель справляется с исправлением ошибок.

\section{Предложенный метод}

Модификация, которая была применена к нейросетевому приёмнику, основана на статье "CBAM: Convolutional Block Attention Module" (\citet{woo2018cbam}) и направлена на улучшение работы модели за счет добавления механизма внимания.

\begin{figure}[H]
    \centering \includegraphics[width=0.8\textwidth]{CBAM.png}
    \caption{Схема архитектуры CBAM модуля} 
    \label{fig:cbam} 
\end{figure}

\subsection{Исходный нейросетевой приёмник}

Исходный нейросетевой приёмник можно описать как композицию последовательных преобразований. Пусть входные данные $\mathbf{Y} \in \mathbb{C}^{B \times N_r \times N_s \times N_c}$, где $B$ -- размер батча, $N_r$ -- количество приёмных антенн, $N_s$ -- количество OFDM-символов, $N_c$ -- количество поднесущих.

Преобразование входных данных в вещественнозначный тензор:
\begin{equation}
\mathbf{Z}_0 = \text{Conv2D}\left(\text{concat}\left[\Re(\mathbf{Y}^T), \Im(\mathbf{Y}^T), \mathbf{N}_0\right]\right)
\end{equation}
где $\mathbf{Y}^T$ -- транспонированная версия $\mathbf{Y}$ с размещением размерности антенн последней, $\Re(\cdot)$ и $\Im(\cdot)$ -- операторы взятия действительной и мнимой части комплексного числа соответственно, $\mathbf{N}_0 = \log_{10}(\mathbf{n}_0) \otimes \mathbf{1}_{N_s \times N_c}$ -- расширенный тензор дисперсии шума.

Далее применяется последовательность остаточных блоков:
\begin{equation}
\mathbf{Z}_l = \mathcal{R}_l(\mathbf{Z}_{l-1}) + \mathbf{Z}_{l-1}, \quad l = 1,\dots,4
\end{equation}
где $\mathcal{R}_l$ -- $l$-й остаточный блок, определяемый как:
\begin{align}
\mathbf{Z}_l^{(1)} &= \text{ReLU}(\text{LayerNorm}(\mathbf{Z}_{l-1})) \\
\mathbf{Z}_l^{(2)} &= \text{Conv2D}(\mathbf{Z}_l^{(1)}) \\
\mathbf{Z}_l^{(3)} &= \text{ReLU}(\text{LayerNorm}(\mathbf{Z}_l^{(2)})) \\
\mathbf{Z}_l^{(4)} &= \text{Conv2D}(\mathbf{Z}_l^{(3)}) \\
\mathcal{R}_l(\mathbf{Z}_{l-1}) &= \mathbf{Z}_l^{(4)}
\end{align}

Итоговые логарифмические отношения правдоподобия (англ. LLR) вычисляются как:
\begin{equation}
\mathbf{LLR} = \text{Conv2D}(\mathbf{Z}_4)
\end{equation}

\subsection{Модификация с механизмом внимания CBAM}

Модификация заключается в замене стандартных остаточных блоков $\mathcal{R}_l$ на блоки с двойным механизмом внимания CBAM. Новый остаточный блок $\mathcal{R}_l^{CBAM}$ определяется как:

\begin{align}
\mathbf{Z}_l^{(1)} &= \text{ReLU}(\text{LayerNorm}(\mathbf{Z}_{l-1})) \\
\mathbf{Z}_l^{(2)} &= \text{Conv2D}(\mathbf{Z}_l^{(1)}) \\
\mathbf{Z}_l^{(3)} &= \text{ReLU}(\text{LayerNorm}(\mathbf{Z}_l^{(2)})) \\
\mathbf{Z}_l^{(4)} &= \text{Conv2D}(\mathbf{Z}_l^{(3)}) \\
\mathbf{Z}_l^{(5)} &= \mathcal{C}(\mathbf{Z}_l^{(4)}) \quad \text{(канальное внимание)} \\
\mathbf{Z}_l^{(6)} &= \mathcal{S}(\mathbf{Z}_l^{(5)}) \quad \text{(пространственное внимание)} \\
\mathcal{R}_l^{CBAM}(\mathbf{Z}_{l-1}) &= \mathbf{Z}_l^{(6)}
\end{align}

\begin{figure}[H]
    \centering \includegraphics[width=1.0\textwidth]{CBAM_NR.png}
    \caption{Схема модифицированного нейросетевого приёмника} 
\end{figure}

\subsubsection{Канальное внимание $\mathcal{C}$}

Канальное внимание вычисляет веса для каждого канала признакового пространства:
\begin{align}
\mathbf{A}_{avg} &= \text{MLP}\left(\frac{1}{H \cdot W}\sum_{i=1}^{H}\sum_{j=1}^{W} \mathbf{Z}[:,i,j,:]\right) \\
\mathbf{A}_{max} &= \text{MLP}\left(\max_{i,j} \mathbf{Z}[:,i,j,:]\right) \\
\mathcal{C}(\mathbf{Z}) &= \mathbf{Z} \otimes \sigma(\mathbf{A}_{avg} + \mathbf{A}_{max})
\end{align}
где $\sigma$ -- сигмоидная функция активации, $\otimes$ -- поэлементное умножение с broadcast по пространственным измерениям, MLP -- двухслойный перцептрон со скрытым слоем размерности $C/r$.

\subsubsection{Пространственное внимание $\mathcal{S}$}

Пространственное внимание вычисляет веса для каждой пространственной позиции:
\begin{align}
\mathbf{S}_{avg} &= \text{mean}_{channel}(\mathbf{Z}) \\
\mathbf{S}_{max} &= \text{max}_{channel}(\mathbf{Z}) \\
\mathcal{S}(\mathbf{Z}) &= \mathbf{Z} \otimes \text{Conv2D}_{7\times7}(\text{concat}[\mathbf{S}_{avg}, \mathbf{S}_{max}])
\end{align}
где свертка с ядром $7\times7$ имеет один выходной канал и сигмоидную активацию.

\subsection{Итоговый модифицированный приёмник}

Модифицированный нейросетевой приёмник описывается следующими преобразованиями:

\begin{equation}
\mathbf{Z}_0 = \text{Conv2D}\left(\text{concat}\left[\Re(\mathbf{Y}^T), \Im(\mathbf{Y}^T), \mathbf{N}_0\right]\right)
\end{equation}

\begin{equation}
\mathbf{Z}_l = \mathcal{R}_l^{CBAM}(\mathbf{Z}_{l-1}) + \mathbf{Z}_{l-1}, \quad l = 1,\dots,4
\end{equation}

\begin{equation}
\mathbf{LLR} = \text{Conv2D}(\mathbf{Z}_4)
\end{equation}

где $\mathcal{R}_l^{CBAM}$ -- остаточные блоки с механизмом CBAM, последовательно применяющие канальное и пространственное внимание для адаптивного уточнения признакового пространства.

\section{Эксперименты}

\subsection{Описание методов}
\begin{enumerate}
    \item \textbf{Perfect CSI приёмник:} Является теоретическим эталоном. Предполагает, что приёмнику точно известна матрица канала, что позволяет LMMSE equalizer оптимально компенсировать искажения. В реальных условиях не реализуем, но задаёт верхнюю границу производительности.
    \item \textbf{LS estimation приёмник:} Реалистичный подход, в котором для оценки канала используются пилот-символы и LS channel estimator. Оценки с пилотов интерполируются на все поднесущие. Метод прост в реализации, но чувствителен к шуму, особенно в условиях низкого отношения сигнал/шум (англ. SNR), что ухудшает качество эквализации.
    \item \textbf{Нейросетевой приёмник (англ. Neural Receiver) от NVIDIA:} Использует сквозную архитектуру на основе свёрточной нейросети с остаточными блоками, которых 4 или 6. Приёмник напрямую преобразует принятую OFDM-решётку в LLR, минуя явные этапы оценки канала и эквализации. Обучен адаптироваться к характеристикам канала, но требует предварительного обучения под конкретные условия.
    \item \textbf{Нейросетевой приёмник с CBAM Residual Blocks:} Модифицированная версия нейросетевого приёмника, в которой стандартные остаточные блоки заменены на блоки с механизмом двойного внимания CBAM. Данная архитектура позволяет сети адаптивно выделять наиболее информативные каналы и пространственные области признакового пространства, улучшая качество детектирования за счет подавления шумовых компонент и усиления значимых признаков. Сохраняет преимущества сквозного подхода, но обладает повышенной способностью к фильтрации помех.
\end{enumerate}

\subsection{Сравнение результатов}


\begin{table}[H]
\centering
\resizebox{\textwidth}{!}{
\begin{tabular}{|c|c|c|c|c|}
\hline
\textbf{Channel} & \textbf{BLER} & \textbf{\shortstack{Neural Receiver\\with 4 Residual\\Blocks (dB)}} & \textbf{\shortstack{Neural Receiver\\with 6 Residual\\Blocks (dB)}} & \textbf{\shortstack{Neural Receiver\\with 4 CBAM\\Residual Blocks (dB)}} \\ \hline
\multirow{2}{*}{A} & $10^{-1}$ & \cellcolor{red!30}\textbf{0.345} & 0.276 & \cellcolor{green!30}\textbf{0.207} \\ \cline{2-5}
                   & $10^{-2}$ & \cellcolor{red!30}\textbf{0.483} & 0.345 & \cellcolor{green!30}\textbf{0.241} \\ \hline
\multirow{2}{*}{B} & $10^{-1}$ & \cellcolor{red!30}\textbf{0.492} & 0.431 & \cellcolor{green!30}\textbf{0.308} \\ \cline{2-5}
                   & $10^{-2}$ & \cellcolor{red!30}\textbf{0.799} & 0.738 & \cellcolor{green!30}\textbf{0.431} \\ \hline
\multirow{2}{*}{C} & $10^{-1}$ & \cellcolor{red!30}\textbf{0.345} & \cellcolor{red!30}\textbf{0.345} & \cellcolor{green!30}\textbf{0.276} \\ \cline{2-5}
                   & $10^{-2}$ & 0.345 & \cellcolor{red!30}\textbf{0.414} & \cellcolor{green!30}\textbf{0.276} \\ \hline
\multirow{2}{*}{D} & $10^{-1}$ & 0.207 & \cellcolor{red!30}\textbf{0.345} & \cellcolor{green!30}\textbf{0.172} \\ \cline{2-5}
                   & $10^{-2}$ & 0.310 & \cellcolor{red!30}\textbf{0.621} & \cellcolor{green!30}\textbf{0.207} \\ \hline
\multirow{2}{*}{E} & $10^{-1}$ & 0.207 & \cellcolor{red!30}\textbf{0.414} & \cellcolor{green!30}\textbf{0.138} \\ \cline{2-5}
                   & $10^{-2}$ & 0.241 & \cellcolor{red!30}\textbf{0.621} & \cellcolor{green!30}\textbf{0.138} \\ \hline
\end{tabular}
}
\caption{Детальное сравнение эффективности различных архитектур нейросетевых приёмников относительно эталонного приёмника с Perfect CSI. В таблице представлены значения разницы в [дБ] между рассматриваемыми нейросетевыми приёмниками и идеальным приёмником для пяти различных типов каналов CDL (A-E) при двух фиксированных значениях BLER: $10^{-1}$ и $10^{-2}$. Каждый столбец соответствует определённой архитектуре: базовый нейросетевой приёмник с 4 остаточными блоками, улучшенная версия с 6 остаточными блоками и предложенная модификация с 4 CBAM Residual Blocks. Зелёным цветом выделены наилучшие результаты (минимальное отклонение от идеального приёмника), красным — наихудшие результаты (максимальное отклонение). Анализ данных демонстрирует, что архитектура с CBAM Residual Blocks последовательно показывает наилучшую производительность среди всех типов каналов и значений BLER, что подтверждает эффективность механизма внимания для задач детектирования в OFDM-системах.}
\label{tab:comparison}
\end{table}

В качесте экспериментов проведено сравнение модифицированного нейросетевого приёмника с тремя ранее рассмотренными моделями: Perfect CSI, LS estimation и нейросетевым приёмником от NVIDIA. Сравнение проведено на всех видах CDL каналов: A , B , C , D , E .   

В CDL каналах модели A, B и C относятся к условиям вне зоны прямой видимости (англ. Non-Line-of-Sight, NLOS), а D и E — к условиям в пределах прямой видимости (англ. Line-of-Sight, LOS). 
NLOS подразумевает, что сигнал достигает приемника через отражения, дифракцию или рассеивание из-за препятствий, таких как здания, деревья или другие объекты. 
NLOS-каналы (A, B, C) характеризуются более высокими потерями на трассе, большей задержкой распространения и выраженным многолучевым эффектом, что приводит к межсимвольной интерференции. В то же время LOS означает, что между передатчиком и приемником существует прямая видимость, без физических препятствий. LOS-каналы (D, E) обладают более стабильными характеристиками, меньшими задержками и преобладанием основного луча, что улучшает качество сигнала. 

\begin{figure}[H]
    \centering
    \begin{subfigure}{0.49\textwidth}
        \centering
        \includegraphics[width=1.0\textwidth]{result_A (4).pdf}
        \caption{CDL канал A (NLOS)}
        \label{fig:channel_B}
    \end{subfigure}

    \vspace{0.7cm}
    
    \begin{subfigure}{0.49\textwidth}
        \centering
        \includegraphics[width=1.0\textwidth]{result_B (1).pdf}
        \caption{CDL канал B (NLOS)}
        \label{fig:channel_B}
    \end{subfigure}
    \hfill
    \begin{subfigure}{0.49\textwidth}
        \centering
        \includegraphics[width=1.0\textwidth]{result_C (4) (2).pdf}
        \caption{CDL канал C (NLOS)}
        \label{fig:channel_C}
    \end{subfigure}
    
    \vspace{0.7cm}
    
    \begin{subfigure}{0.49\textwidth}
        \centering
        \includegraphics[width=1.0\textwidth]{result_D.pdf}
        \caption{CDL канал D (LOS)}
        \label{fig:channel_D}
    \end{subfigure}
    \hfill
    \begin{subfigure}{0.49\textwidth}
        \centering
        \includegraphics[width=1.0\textwidth]{result_E (2).pdf}
        \caption{CDL канал E (LOS)}
        \label{fig:channel_E}
    \end{subfigure}
    \caption{Сравнение производительности различных приёмных систем на пяти типах CDL каналов. На каждом графике представлены пять кривых, отображающих зависимость BLER от отношения сигнал/шум $E_b/N_0$ [дБ] для: эталонного приёмника с Perfect CSI (идеальная информация о канале), классического приёмника с LS-оценкой канала, нейросетевого приёмника с 4 остаточными блоками, нейросетевого приёмника с 6 остаточными блоками и предложенного нейросетевого приёмника с 4 CBAM Residual Blocks. Графики демонстрируют, что архитектура с CBAM Residual Blocks последовательно приближается к производительности идеального приёмника, превосходя как классические методы, так и базовые нейросетевые архитектуры. Логарифмическая шкала по оси ординат позволяет наглядно оценить экспоненциальное снижение вероятности ошибок при улучшении условий приёма.}
    \label{fig:all_channels}
\end{figure}

\bigskip
В итоге, из графиков и таблицы сравнения приёмников на каждой модели CDL канала можно заметить следующее:

\begin{enumerate}
    \item Простой подход с увеличением числа Residual блоков привёл к незначительным улучшениям нейросетевого приёмника на моделях A и B (улучшения до 0,138 [дБ]), а на остальных моделях CDL канала продемонстрированы ухудшения показателей (ухудшение до 0,380 [дБ]).
    \item Предложенный метод модификации нейросетевого приёмника оказался действительно рабочим и продемонстрировал значимые результаты на всех моделях CDL канала: заметен явный прирост в качестве в сравнении с исходным нейросетевым приёмником от NVIDIA (улучшение на NLOS каналах в диапазоне от 0.069 [дБ] до 0.368 [дБ] и улучшение на LOS каналах в диапазоне от 0.034 [дБ] до 0.103 [дБ]), а на CDL-E кривая модифицированной модели почти аппроксимиреут кривую идеального приёмника с разницей на 0.138 [дБ].
\end{enumerate}

\bigskip
\section{Заключение}

В статье проведено исследование применения методов глубокого обучения для оптимизации PHY Layer в системах связи 5G. Рассмотрены как классические подходы к обработке сигналов (LS channel estimator, LMMSE equalizer), так и современный нейросетевой приёмник на основе остаточных свёрточных блоков. Показано, что базовая нейросетевая архитектура демонстрирует значительное превосходство над традиционными методами, приближаясь к эталонному сценарию с полным знанием канала (Perfect CSI).

Основной вклад работы — разработка модифицированного нейросетевого приёмника, усиленного механизмом внимания CBAM. Экспериментальная оценка на наборе стандартизированных моделей каналов CDL подтвердила эффективность предложенного подхода. Модифицированный приёмник показал устойчивое улучшение помехоустойчивости по сравнению с базовой нейросетевой архитектурой, особенно в сложных условиях NLOS-распространения.

Практическая значимость исследования заключается в демонстрации потенциала механизмов внимания для создания более эффективных и адаптивных систем приёма данных. Предложенное решение открывает пути для дальнейшей оптимизации нейросетевых алгоритмов в целях повышения спектральной эффективности и надёжности беспроводных сетей.

\clearpage
\bibliographystyle{plainnat}
\bibliography{references}


\end{document}