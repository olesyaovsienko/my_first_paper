\documentclass{article}
\usepackage{arxiv}

\usepackage[utf8]{inputenc}
\usepackage[english, russian]{babel}
\usepackage[T1]{fontenc}
\usepackage{url}
\usepackage{booktabs}
\usepackage{amsfonts}
\usepackage{nicefrac}
\usepackage{microtype}
\usepackage{lipsum}
\usepackage{graphicx}
\usepackage{natbib}
\usepackage{doi}

\usepackage[warn]{mathtext}
\usepackage[T2A]{fontenc}			% кодировка
\usepackage[utf8]{inputenc}			% кодировка исходного текста
\usepackage[english,russian]{babel}	% локализация и переносы
\usepackage{indentfirst}
\usepackage{csquotes}
% \usepackage[bibstyle=gost-numeric, sorting=none]{biblatex}
% \addbibresource{biblio.bib}

\linespread{1.3}

\usepackage[utf8]{inputenc}
\usepackage[T2A]{fontenc}
\usepackage[russian]{babel}
\usepackage{hyphenat} 

\usepackage{array}
\usepackage{multirow}
\usepackage{xcolor}
\usepackage{colortbl}

% page settings
\usepackage[
    left=1.5cm,
    right=1.5cm,
    top=1.5cm,
    bottom=1.5cm,
    bindingoffset=0cm
]{geometry}

\usepackage{graphicx, hyperref, xcolor}
\hypersetup{
    colorlinks=true,
    linkcolor=teal,
    filecolor=magenta, 
    urlcolor=blue,
    citecolor=olive,
    pdftitle={GD},
    % pdfpagemode=FullScreen,
    linktoc=all
    }

\usepackage{wrapfig,caption}

% figures
\usepackage{caption}
\usepackage{subcaption}
\usepackage{floatrow}
\floatsetup{heightadjust=object}

% math
\usepackage{amsmath,amsfonts,amssymb,amsthm,mathtools,esint,eucal}


\def\vec#1{\mathchoice{\mbox{\boldmath$\displaystyle#1$}}
{\mbox{\boldmath$\textstyle#1$}} {\mbox{\boldmath$\scriptstyle#1$}} {\mbox{\boldmath$\scriptscriptstyle#1$}}}

\title{Применение машинного обучения в 5G. Нейросетевой приёмник и его CBAM модификация.}

\author{Бобров Е. А.\\
	МГУ им. М.В. Ломоносова\\
	факультет ВМК\\
        119991, Москва, Ленинские горы, д.1, стр. 52\\
        \And
        Овсиенко О. П. \\
	МГУ им. М.В. Ломоносова\\
	факультет ВМК\\
        119991, Москва, Ленинские горы, д.1, стр. 52\\
	\texttt{st02220041@gse.cs.msu.ru} \\
}
\date{}

\renewcommand{\shorttitle}{\textit{arXiv} Template}

\hypersetup{
pdftitle={My first paper Ovsienko},
pdfkeywords={neural receiver, CBAM, 5G},
}

\begin{document}
\maketitle

\begin{abstract}
В статье исследуется применении методов машинного и глубокого обучения для оптимизации физического уровня (PHY Layer) в 5G-сетях. Рассмотрены традиционные методы обработки сигналов, такие как LMMSE-эквализация и LS-оценка канала, а также нейросетевые подходы, включая архитектуру с остаточными блоками от NVIDIA. Центральным результатом работы стала разработка модифицированного нейросетевого приёмника с механизмом внимания CBAM, который продемонстрировал превосходство над классическими методами в условиях многолучевого распространения и помех. Практическая значимость исследования заключается в возможности интеграции предложенного метода в современные системы связи для повышения их эффективности и устойчивости к помехам.
\end{abstract}

\keywords{традиционные методы обработки сигналов\and нейросетевой приёмник}

\section*{1) Введение}

Современные системы мобильной связи пятого поколения (5G) сталкиваются с необходимостью обеспечения исключительно высокой скорости передачи данных, сверхнизких задержек и надёжного соединения для миллионов устройств. Эти требования предъявляют новые вызовы к методам обработки сигналов на физическом уровне (PHY Layer), особенно в условиях многолучевого распространения и интерференции. Традиционные алгоритмы, такие как оценка канала методом наименьших квадратов (LS) и линейная эквализация (LMMSE), зачастую не справляются с растущей сложностью каналов, что стимулирует поиск более совершенных и адаптивных подходов.

Классические подходы к обработке сигналов в системах OFDM и MIMO, такие как оценка канала на основе пилот-сигналов и последующая LMMSE-эквализация, хорошо изучены и обеспечивают приемлемое качество в условиях умеренных помех. Однако их эффективность существенно снижается в сценариях с сильной интерференцией, быстрыми замираниями и в отсутствие точной информации о состоянии канала (Channel State Information, CSI). Метод наименьших квадратов (LS), будучи вычислительно простым, крайне чувствителен к шуму, в то время как более робастные методы, такие как LMMSE, требуют априорного знания статистики канала и шума, что зачастую недостижимо в реальных динамичных условиях.

В последние годы активно развиваются методы машинного и глубокого обучения, демонстрирующие значительный потенциал для оптимизации телекоммуникационных систем. В частности, нейросетевые приёмники, способные обучаться сквозному преобразованию принятого сигнала в битовые последовательности, показывают результаты, превосходящие классические методы. Одной из таких современных архитектур является нейросетевой приёмник от NVIDIA, построенный на основе остаточных свёрточных блоков (Residual Blocks), который позволяет эффективно компенсировать искажения сигнала без явного знания матрицы канала.

В данной работе предлагается модификация данного нейросетевого приёмника, направленная на дальнейшее повышение его эффективности. Модификация заключается во внедрении механизма внимания CBAM (Convolutional Block Attention Module) в структуру остаточных блоков. Этот механизм позволяет сети адаптивно выделять наиболее информативные признаки в частотно-временном и канальном пространствах, усиливая полезные компоненты сигнала и подавляя шумовые.

Экспериментальные исследования, проведённые на стандартизированных моделях каналов CDL (Clustered Delay Line), подтвердили эффективность предложенного подхода. Модифицированный приёмник продемонстрировал устойчивое улучшение показателя BLER (Block Error Rate) по сравнению как с классическими методами (LS, LMMSE), так и с базовой нейросетевой архитектурой, в особенности в условиях нелинейной видимости (NLOS) и при низких отношениях сигнал/шум. Таким образом, вклад работы заключается в разработке и валидации усовершенствованной архитектуры приёмника, что открывает пути для создания более помехоустойчивых и спектрально-эффективных систем связи следующего поколения.

\section*{2) Обзор литературы}

Современные системы связи пятого поколения требуют принципиально новых подходов к обработке сигналов на физическом уровне. Традиционные методы, такие как оценка канала и эквализация, хорошо изучены, но имеют существенные ограничения в условиях сложных сценариев распространения сигнала. \citet{badr2019lmmse} проводит всесторонний анализ LMMSE-оценки канала в контексте OFDM систем, подчеркивая её преимущества перед более простыми методами, такими как метод наименьших квадратов (LS), но также отмечая зависимость от точного знания статистических характеристик канала. Исследование \citet{bagadi2010mimo} демонстрирует применение пилот-сигналов для оценки канала в системах MIMO-OFDM, что является стандартным подходом в современных системах связи.

Архитектуры MIMO систем играют ключевую роль в повышении пропускной способности и надежности беспроводных каналов. \citet{shah2017performance} предоставляет сравнительный анализ различных конфигураций MIMO, включая SISO, SIMO, MISO и MIMO, показывая преимущества пространственного разнесения и мультиплексирования. Дальнейшее развитие MIMO технологий представлено в работе \citet{pessoa2020cdl}, где предлагается модель канала на основе CDL с двухполяризованными антеннами, специально адаптированная для сценариев 5G в сельской местности.

Теоретические основы достижимых скоростей передачи в системах связи продолжают развиваться. \citet{bjornson2015achievable} исследует предельные возможности Rician MIMO каналов с учетом аппаратных искажений, что особенно актуально для систем массового MIMO. В то же время \citet{caire2017achievable} фокусируется на методах вероятностного формирования сигналов (probabilistic shaping) для повышения спектральной эффективности.

Прорыв в области глубокого обучения открыл новые перспективы для оптимизации физического уровня. Архитектура глубоких остаточных свёрточных сетей, представленная в работе \citet{he2016deep}, стала фундаментом для многих последующих разработок, включая нейросетевые приемники. Механизмы внимания, такие как CBAM (Convolutional Block Attention Module) из исследования \citet{woo2018cbam}, демонстрируют эффективность в выделении информативных признаков в компьютерном зрении, что может быть успешно перенесено на задачи обработки сигналов.

Современные инструменты для исследований в области связи, такие как Sionna от NVIDIA \citep{sionna}, предоставляют мощную платформу для сквозного моделирования систем связи и интеграции методов машинного обучения. В частности, \citet{sionna2023neural} демонстрирует применение нейросетевых приемников для OFDM SIMO систем. Исследования \citet{levis2020deeprx} и \citet{levis2020end} развивают концепцию полностью сверточного глубокого приемника и сквозного обучения для OFDM систем, включая возможность беспилотной коммуникации.

Важным аспектом современных систем связи являются эффективные методы кодирования. Работы \citet{nguyen2019efficient} и \citet{nguyen2021low} посвящены разработке низкосложных декодеров QC-LDPC для 5G New Radio, что критически важно для практической реализации систем следующего поколения. Перспективным направлением является также совместное демодулирование сложных созвездий с использованием машинного обучения, как показано в исследовании \citet{gansekoele2025joint}.

Таким образом, современное состояние исследований демонстрирует четкий тренд к интеграции методов машинного обучения с классическими подходами обработки сигналов, что открывает новые возможности для создания более эффективных и устойчивых систем связи.

\section*{3) Математическая постановка задачи}
Рассматривается система MIMO с \( N_\text{t} \) передающими антеннами и \( N_\text{r} \) принимающими антеннами. Пусть:
\begin{equation}
\mathbf{x} = \mathbf{W} \mathbf{s}
\end{equation}
где \(\mathbf{W} \in \mathbb{C}^{N_\text{t} \times N_\text{r}}\) – матрица прекодера, \(\mathbf{s} \in \mathbb{C}^{N_\text{r} \times 1}\) – вектор исходных сигналов ( бинарная последовательность ), а \(\mathbf{x} \in \mathbb{C}^{N_\text{t} \times 1}\) - передаваемый вектор сигналов после прекодера (каждый символ в пропорции распространён по всем антеннам). Этот вектор обычно берётся из некоторой M-QAM/QPSK сигнальной диаграммы, т.\,е. \( \mathbf{x} \) состоит из комплексных значений, отражающих точку созвездия.

Модель прохождения сигнала через канал можно записать в следующем виде:

\begin{equation}
\mathbf{y} = \mathbf{H} \mathbf{x} + \mathbf{n},
\end{equation}
где
\[
\mathbf{y} \in \mathbb{C}^{N_\text{r} \times 1}
\]
– вектор принятых сигналов на антенных элементах приёмника, \(\mathbf{H} \in \mathbb{C}^{N_\text{r} \times N_\text{t}}\) – матрица канала, описывающая затухание и фазовые сдвиги на пути «от каждой передающей антенны к каждой приёмной», а \(\mathbf{n} \in \mathbb{C}^{N_\text{r} \times 1}\) – вектор аддитивного белого гауссовского шума (AWGN) с дисперсией \(\sigma_n^2\).

В классической постановке задачи \textbf{MIMO-декодирования} (или «символьной детекции») требуется найти оценки \(\hat{\mathbf{x}}\) переданных символов, обладающие минимальной вероятностью ошибки. При стандартном подходе (например, методе MMSE – Minimum Mean Square Error) оценивается:
\begin{equation}
\hat{\mathbf{x}}_\mathrm{MMSE} = \underset{\mathbf{x}}{\mathrm{argmin}}( 
\Big\lVert \mathbf{y} - \mathbf{H}\mathbf{x} \Big\rVert^2 + \sigma_n^2 \lVert \mathbf{x} \rVert^2).
\end{equation}
Для высоких размерностей (большое число антенн \(N_\text{t}\) и \(N_\text{r}\)), а также для сложных созвездий (например, 64-QAM, 256-QAM) точный поиск \(\mathbf{x}\) может становиться вычислительно дорогостоящим. Методы глубокого обучения позволяют решать данную задачу итеративно или напрямую, обучаясь на большом наборе «принятый сигнал – истинный символ» без явной оптимизации по формуле минимума ошибки.

В рамках предлагаемой работы рассматривается ситуация, когда матрица канала \(\mathbf{H}\) частично неизвестна или может изменяться со временем, а число передатчиков и приёмников велико. Требуется реализовать «глубокую» нейронную сеть, которая сможет, во-первых, оценивать параметры \(\mathbf{H}\) по пилотным символам, а во-вторых, автоматически восстанавливать \(\mathbf{x}\) из \(\mathbf{y}\) с учётом сложности канала и помех.

\textbf{Основная цель} формализуется как задача:
\begin{equation}
\hat{\mathbf{x}} = \mathcal{F}_\theta(\mathbf{y}, \mathbf{p}),
\end{equation}
где \(\mathbf{p}\) – набор пилотных символов, а \(\mathcal{F}_\theta\) – функция, задаваемая параметрами нейронной сети \(\theta\). 

Требуется найти такие параметры \(\theta\), которые минимизируют среднюю вероятность ошибки детектирования:
\begin{equation}
\theta^* = \underset{\theta}{\mathrm{argmin}} \; \mathbb{E}_{(\mathbf{x},\mathbf{H},\mathbf{n})} 
\Big\{ \ell\big(\mathcal{F}_\theta(\mathbf{y}, \mathbf{p}),\, \mathbf{x}\big) \Big\},
\end{equation}
где \(\ell(\cdot)\) – функция потерь (кросс-энтропия).

Для оценки качества декодирования в моделях связи используются BLER и $\frac{E_b}{N_0}$: 

\begin{equation}
        \text{BLER} = \frac{\text{Количество ошибочно переданных блоков}}{\text{Общее количество переданных блоков}}\text{ , }
        \end{equation}
        \begin{equation}
        \frac{E_b}{N_0} = \frac{\text{Энергия на один бит}}{\text{Спектральная плотность мощности шума}} \text{  [дБ]}
\end{equation}

Ошибочный блок — это блок, в котором хотя бы один бит не совпадает с исходным. BLER помогает отслеживать эффективность модели, показывая, насколько часто возникают ошибки в передаваемых данных. Чем ниже BLER, тем лучше модель справляется с исправлением ошибок.

\bigskip

\section*{4) Базовый эксперимент}
\subsection*{4.1) Описание методов}
\begin{enumerate}
    \item \textbf{Perfect CSI приёмник:} Является теоретическим эталоном (идеальным бейзлайном). Предполагает, что приёмнику точно известна матрица канала, что позволяет LMMSE-эквалайзеру оптимально компенсировать искажения. В реальных условиях не реализуем, но задаёт верхнюю границу производительности.
    \item \textbf{LS estimation приёмник:} Реалистичный подход, в котором для оценки канала используются пилот-символы и метод наименьших квадратов (LS). Оценки с пилотов интерполируются на все поднесущие. Метод прост в реализации, но чувствителен к шуму, особенно в условиях низкого SNR, что ухудшает качество эквализации.
    \item \textbf{Нейросетевой приёмник от NVIDIA:} Использует сквозную архитектуру на основе свёрточной нейросети с остаточными блоками. Приёмник напрямую преобразует принятую OFDM-решётку в LLR, минуя явные этапы оценки канала и эквализации. Обучен адаптироваться к характеристикам канала, но требует предварительного обучения под конкретные условия.
\end{enumerate}

\subsection*{4.2) Сравнение результатов}
Для проведения сопоставления рассмотренных моделей был построен график, который иллюстрирует сравнение трёх различных приёмных систем по показателю BLER (Block Error Rate) в зависимости от отношения сигнал/шум $E_b/N_0$, выраженного в децибелах. На оси абсцисс отложены значения $E_b/N_0$ в [дБ], а на оси ординат — значения BLER в логарифмическом масштабе. Это позволяет наглядно проследить, как снижается вероятность ошибочной передачи блока данных при улучшении условий приёма.

На графике представлены три кривые. Первая — это «Baseline — Perfect CSI», которая отображает производительность системы, обладающей идеальной информацией о канале связи. Эта система служит эталоном, демонстрируя наилучший возможный результат. Вторая кривая — «Baseline — LS Estimation» — показывает результат системы с линейной оценкой канала, которая не обладает идеальной матрицей канала. Третья кривая соответствует «Neural receiver with 4 Residual Blocks» — системе, использующей нейросетевой приёмник, состоящий из четырёх остаточных блоков. Она обучена восстанавливать данные в условиях шумного канала без явной информации о канале.

\begin{figure}[H]
    \includegraphics[width=0.93\textwidth]{result_NVIDIA.pdf}
\end{figure}


По графику видно, что при низких значениях $E_b/N_0$ (до - 2 [дБ]) все системы работают плохо, и BLER близок к 1 — почти все блоки данных передаются с ошибками. Однако по мере увеличения $E_b/N_0$ различия между системами становятся заметными. Система с идеальным знанием канала начинает быстрее снижать BLER, за ней следует нейросетевой приёмник, а LS-оценка отстаёт. Заметим, что нейросетевой приёмник приближается к производительности системы с идеальной CSI, что говорит о его способности эффективно восстанавливать данные даже без точной информации о канале.

Таким образом, график демонстрирует, что нейросетевой приёмник с четырьмя остаточными блоками существенно превосходит традиционный подход с LS-оценкой канала и приближается по эффективности к теоретическому максимуму. Это подчёркивает потенциал нейросетевых методов в задачах приёма и декодирования данных в системах связи.

\section*{5) Предложенный метод}

Была поставлена задача: усовершенствовать нейросетевой приёмник и достичь лучших результатов при помощи своей модификации. 

Очевидной первой идеей по возможной модификации нейронной сети было добавление дополнительных слоёв. Вместо четырёх Residual блоков было использовано шесть, но такой подход усовершенствования нейросетевого приёмника не привёл к значимому улучшению показателей, а в некоторых случаях даже ухудшил их (что продемонстрировано на графиках далее).

Модификация, которая была применена к нейросетевому приёмнику, основана на статье "CBAM: Convolutional Block Attention Module" (\citet{woo2018cbam}) и направлена на улучшение работы модели за счет добавления механизма внимания.

\begin{figure}[H]
    \centering \includegraphics[width=0.8\textwidth]{CBAM.png}
    \caption{Схема архитектуры CBAM модуля} 
    \label{fig:cbam} 
\end{figure}

Основная идея CBAM заключается в последовательном применении двух типов внимания — канального (channel attention) и пространственного (spatial attention) — для адаптивного уточнения признаков на промежуточных этапах обработки. Это позволяет сети сосредотачиваться на наиболее информативных частях входных данных, подавляя при этом менее значимые элементы.

В итоге, стандартные остаточные блоки (ResidualBlock) были заменены на блоки с механизмом внимания CBAM (CBAMResidualBlock). Каждый такой блок теперь включает в себя два основных компонента: модуль канального внимания (ChannelAttention) и модуль пространственного внимания (SpatialAttention).


\begin{figure}[H]
    \centering \includegraphics[width=0.85\textwidth]{CA.png}
    \caption{Схема архитектуры канального внимания} 
\end{figure}

Модуль канального внимания вычисляет важность каждого канала в признаковом пространстве, используя как среднее, так и максимальное значения по пространственным измерениям (временным и частотным). Эти значения пропускаются через общую многослойную перцептронную сеть (MLP), что позволяет получить веса для каждого канала. Эти веса применяются к исходным признакам, усиливая значимые каналы и ослабляя менее важные. 

\begin{figure}[H]
    \centering \includegraphics[width=0.9\textwidth]{SA.png}
    \caption{Схема архитектуры пространственного внимания} 
\end{figure}

Модуль пространственного внимания, в свою очередь, фокусируется на пространственных областях признакового пространства. Он объединяет средние и максимальные значения по каналам, а затем применяет свертку с ядром 7x7 для получения карты пространственного внимания. Эта карта умножается на признаки, выделяя наиболее информативные области.

\begin{figure}[H]
    \centering \includegraphics[width=1.0\textwidth]{CBAM_NR.png}
    \caption{Схема модифицированного нейросетевого приёмника} 
\end{figure}


Важно отметить, что модули внимания добавляются после сверточных слоев в остаточном блоке, но перед skip-connection. Это позволяет сети адаптивно корректировать признаки перед их суммированием с исходными данными. Такая последовательность действий соответствует рекомендациям из статьи, где показано, что канальное внимание, примененное перед пространственным, дает наилучшие результаты. Кроме того, была сохранена исходная структура нормализации слоев и активаций (ReLU), что обеспечивает стабильность обучения.

Таким образом, модифицированный нейросетевой приёмник теперь состоит из входной свертки, четырех остаточных блоков с CBAM и выходной свертки. Это позволяет сети не только извлекать признаки из входных данных, но и адаптивно уточнять их, учитывая как канальные, так и пространственные зависимости. Такое усовершенствование, как показано в статье, может привести к улучшению точности классификации и детекции, поскольку сеть учится более эффективно использовать информативные признаки и игнорировать шум или менее значимые детали. 

\section*{6) Эксперименты}

В качесте экспериментов проведено сравнение модифицированного нейросетевого приёмника с тремя ранее рассмотренными моделями: Perfect CSI, LS estimation и нейросетевым приёмником от NVIDIA. Сравнение проведено на всех видах CDL каналов: A , B , C , D , E .  

CDL (Clustered Delay Line) — это модель канала, используемая в стандартах 3GPP для имитации многолучевого распространения сигнала в беспроводных системах связи, таких как 5G NR (New Radio). Она основана на принципе кластеров — групп лучей с близкими временными задержками и углами прихода/ухода, что позволяет реалистично моделировать различные сценарии передачи сигнала в разных условиях (LOS/NLOS, городская среда, сельская местность и т. д.).  

В CDL каналах модели A, B и C относятся к NLOS-условиям (Non-Line-of-Sight), а D и E — к LOS (Line-of-Sight). 
NLOS подразумевает, что сигнал достигает приемника через отражения, дифракцию или рассеивание из-за препятствий, таких как здания, деревья или другие объекты. 
NLOS-каналы (A, B, C) характеризуются более высокими потерями на трассе, большей задержкой распространения и выраженным многолучевым эффектом, что приводит к межсимвольной интерференции. В то же время LOS означает, что между передатчиком и приемником существует прямая видимость, без физических препятствий. LOS-каналы (D, E) обладают более стабильными характеристиками, меньшими задержками и преобладанием основного луча, что улучшает качество сигнала. 

\begin{table}[H]
\centering
\resizebox{\textwidth}{!}{
\begin{tabular}{|c|c|c|c|c|>{\columncolor{yellow!30}}c|}
\hline
\textbf{Channel} & \textbf{BLER} & \textbf{\shortstack{Neural Receiver\\with 4 Residual\\Blocks (dB)}} & \textbf{\shortstack{Neural Receiver\\with 6 Residual\\Blocks (dB)}} & \textbf{\shortstack{Neural Receiver\\with 4 CBAM\\Residual Blocks (dB)}} & \textbf{\shortstack{Gain of proposed CBAM modification\\over NVIDIA baseline (dB)}} \\ \hline
\multirow{2}{*}{A} & $10^{-1}$ & \cellcolor{red!30}\textbf{0.345} & 0.276 & \cellcolor{green!30}\textbf{0.207} & \textbf{0.138} \\ \cline{2-6}
                   & $10^{-2}$ & \cellcolor{red!30}\textbf{0.483} & 0.345 & \cellcolor{green!30}\textbf{0.241} & \textbf{0.241} \\ \hline
\multirow{2}{*}{B} & $10^{-1}$ & \cellcolor{red!30}\textbf{0.492} & 0.431 & \cellcolor{green!30}\textbf{0.308} & \textbf{0.184} \\ \cline{2-6}
                   & $10^{-2}$ & \cellcolor{red!30}\textbf{0.799} & 0.738 & \cellcolor{green!30}\textbf{0.431} & \textbf{0.368} \\ \hline
\multirow{2}{*}{C} & $10^{-1}$ & \cellcolor{red!30}\textbf{0.345} & \cellcolor{red!30}\textbf{0.345} & \cellcolor{green!30}\textbf{0.276} & \textbf{0.069} \\ \cline{2-6}
                   & $10^{-2}$ & 0.345 & \cellcolor{red!30}\textbf{0.414} & \cellcolor{green!30}\textbf{0.276} & \textbf{0.069} \\ \hline
\multirow{2}{*}{D} & $10^{-1}$ & 0.207 & \cellcolor{red!30}\textbf{0.345} & \cellcolor{green!30}\textbf{0.172} & \textbf{0.034} \\ \cline{2-6}
                   & $10^{-2}$ & 0.310 & \cellcolor{red!30}\textbf{0.621} & \cellcolor{green!30}\textbf{0.207} & \textbf{0.103} \\ \hline
\multirow{2}{*}{E} & $10^{-1}$ & 0.207 & \cellcolor{red!30}\textbf{0.414} & \cellcolor{green!30}\textbf{0.138} & \textbf{0.069} \\ \cline{2-6}
                   & $10^{-2}$ & 0.241 & \cellcolor{red!30}\textbf{0.621} & \cellcolor{green!30}\textbf{0.138} & \textbf{0.103} \\ \hline
\end{tabular}
}
\caption[]{\shortstack{Разница в [дБ] между рассматриваемыми нейросетевыми приёмниками\\ и идеальным приёмником на всех видах CDL каналов\\ при фиксированных значениях BLER $10^{-1}$ и $10^{-2}$}}
\label{tab:comparison}
\end{table}


\begin{figure}[H]
    \centering
    \begin{subfigure}{0.49\textwidth}
        \centering
        \includegraphics[width=1.0\textwidth]{result_A (4).pdf}
        \caption{CDL канал A (NLOS)}
        \label{fig:channel_B}
    \end{subfigure}

    \vspace{0.7cm}
    
    \begin{subfigure}{0.49\textwidth}
        \centering
        \includegraphics[width=1.0\textwidth]{result_B (1).pdf}
        \caption{CDL канал B (NLOS)}
        \label{fig:channel_B}
    \end{subfigure}
    \hfill
    \begin{subfigure}{0.49\textwidth}
        \centering
        \includegraphics[width=1.0\textwidth]{result_C (4) (2).pdf}
        \caption{CDL канал C (NLOS)}
        \label{fig:channel_C}
    \end{subfigure}
    
    \vspace{0.7cm}
    
    \begin{subfigure}{0.49\textwidth}
        \centering
        \includegraphics[width=1.0\textwidth]{result_D.pdf}
        \caption{CDL канал D (LOS)}
        \label{fig:channel_D}
    \end{subfigure}
    \hfill
    \begin{subfigure}{0.49\textwidth}
        \centering
        \includegraphics[width=1.0\textwidth]{result_E (2).pdf}
        \caption{CDL канал E (LOS)}
        \label{fig:channel_E}
    \end{subfigure}
\end{figure}

\bigskip
В итоге, из графиков сравнения приёмников на каждой модели CDL канала можно заметить следующее:

\begin{enumerate}
    \item Простой подход с увеличением числа Residual блоков привёл к незначительным улучшениям нейросетевого приёмника на моделях A и B (улучшения до 0,138 [дБ]), а на остальных моделях CDL канала продемонстрированы ухудшения показателей (ухудшение до 0,380 [дБ]).
    \item Предложенный метод модификации нейросетевого приёмника оказался действительно рабочим и продемонстрировал значимые результаты на всех моделях CDL канала: заметен явный прирост в качестве в сравнении с исходным нейросетевым приёмником от NVIDIA (улучшение на NLOS каналах в диапазоне от 0.069 [дБ] до 0.368 [дБ] и улучшение на LOS каналах в диапазоне от 0.034 [дБ] до 0.103 [дБ]), а на CDL-E кривая модифицированной модели почти аппроксимиреут кривую идеального приёмника с разницей на 0.138 [дБ].
\end{enumerate}

Таким образом, благодаря предложенному методу удалось выполнить поставленную задачу и добиться улучшения исходного нейросетевого приёмника от NVIDIA.

\bigskip
\section*{7) Заключение}

В статье проведено исследование применения методов глубокого обучения для оптимизации физического уровня в системах связи 5G. Рассмотрены как классические подходы к обработке сигналов (LS, LMMSE), так и современный нейросетевой приёмник на основе остаточных свёрточных блоков. Показано, что базовая нейросетевая архитектура демонстрирует значительное превосходство над традиционными методами, приближаясь к эталонному сценарию с полным знанием канала (Perfect CSI).

Основной вклад работы — разработка модифицированного нейросетевого приёмника, усиленного механизмом внимания CBAM. Экспериментальная оценка на наборе стандартизированных моделей каналов CDL подтвердила эффективность предложенного подхода. Модифицированный приёмник показал устойчивое улучшение помехоустойчивости (BLER) по сравнению с базовой нейросетевой архитектурой, особенно в сложных условиях NLOS-распространения.

Практическая значимость исследования заключается в демонстрации потенциала механизмов внимания для создания более эффективных и адаптивных систем приёма данных. Предложенное решение открывает пути для дальнейшей оптимизации нейросетевых алгоритмов в целях повышения спектральной эффективности и надёжности беспроводных сетей.

\clearpage
\bibliographystyle{plainnat}
\bibliography{references}


\end{document}