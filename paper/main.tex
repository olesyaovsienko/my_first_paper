\documentclass{article}
\usepackage{arxiv}

\usepackage[utf8]{inputenc}
\usepackage[english, russian]{babel}
\usepackage[T1]{fontenc}
\usepackage{url}
\usepackage{booktabs}
\usepackage{amsfonts}
\usepackage{nicefrac}
\usepackage{microtype}
\usepackage{lipsum}
\usepackage{graphicx}
\usepackage{natbib}
\usepackage{doi}

\usepackage[warn]{mathtext}
\usepackage[T2A]{fontenc}			% кодировка
\usepackage[utf8]{inputenc}			% кодировка исходного текста
\usepackage[english,russian]{babel}	% локализация и переносы
\usepackage{indentfirst}
\usepackage{csquotes}
% \usepackage[bibstyle=gost-numeric, sorting=none]{biblatex}
% \addbibresource{biblio.bib}

\linespread{1.3}

\usepackage[utf8]{inputenc}
\usepackage[T2A]{fontenc}
\usepackage[russian]{babel}
\usepackage{hyphenat} 

\usepackage{array}
\usepackage{multirow}
\usepackage{xcolor}
\usepackage{colortbl}

% page settings
\usepackage[
    left=1.5cm,
    right=1.5cm,
    top=1.5cm,
    bottom=1.5cm,
    bindingoffset=0cm
]{geometry}

\usepackage{graphicx, hyperref, xcolor}
\hypersetup{
    colorlinks=true,
    linkcolor=teal,
    filecolor=magenta, 
    urlcolor=blue,
    citecolor=olive,
    pdftitle={GD},
    % pdfpagemode=FullScreen,
    linktoc=all
    }

\usepackage{wrapfig,caption}

% figures
\usepackage{caption}
\usepackage{subcaption}
\usepackage{floatrow}
\floatsetup{heightadjust=object}

% math
\usepackage{amsmath,amsfonts,amssymb,amsthm,mathtools,esint,eucal}


\def\vec#1{\mathchoice{\mbox{\boldmath$\displaystyle#1$}}
{\mbox{\boldmath$\textstyle#1$}} {\mbox{\boldmath$\scriptstyle#1$}} {\mbox{\boldmath$\scriptscriptstyle#1$}}}

\title{Применение машинного обучения в 5G. Нейросетевой приёмник и его CBAM модификация.}

\author{Бобров Е. А.\\
	МГУ им. М.В. Ломоносова\\
	факультет ВМК\\
        119991, Москва, Ленинские горы, д.1, стр. 52\\
        \And
        Овсиенко О. П. \\
	МГУ им. М.В. Ломоносова\\
	факультет ВМК\\
        119991, Москва, Ленинские горы, д.1, стр. 52\\
	\texttt{st02220041@gse.cs.msu.ru} \\
}
\date{}

\renewcommand{\shorttitle}{\textit{arXiv} Template}

\hypersetup{
pdftitle={My first paper Ovsienko},
pdfkeywords={neural receiver, CBAM, 5G},
}

\begin{document}
\maketitle

\begin{abstract}
В статье исследуется применении методов машинного и глубокого обучения для оптимизации физического уровня (PHY Layer) в 5G-сетях. Рассмотрены традиционные методы обработки сигналов, такие как LMMSE-эквализация и LS-оценка канала, а также нейросетевые подходы, включая архитектуру с остаточными блоками от NVIDIA. Центральным результатом работы стала разработка модифицированного нейросетевого приёмника с механизмом внимания CBAM, который продемонстрировал превосходство над классическими методами в условиях многолучевого распространения и помех. Практическая значимость исследования заключается в возможности интеграции предложенного метода в современные системы связи для повышения их эффективности и устойчивости к помехам.
\end{abstract}

\keywords{традиционные методы обработки сигналов\and нейросетевой приёмник}

\section*{1) Введение}

Современные системы мобильной связи пятого поколения (5G) сталкиваются с необходимостью обеспечения исключительно высокой скорости передачи данных, сверхнизких задержек и надёжного соединения для миллионов устройств. Эти требования предъявляют новые вызовы к методам обработки сигналов на физическом уровне (PHY Layer), особенно в условиях многолучевого распространения и интерференции. Традиционные алгоритмы, такие как оценка канала методом наименьших квадратов (LS) и линейная эквализация (LMMSE), зачастую не справляются с растущей сложностью каналов, что стимулирует поиск более совершенных и адаптивных подходов.

Классические подходы к обработке сигналов в системах OFDM и MIMO, такие как оценка канала на основе пилот-сигналов и последующая LMMSE-эквализация, хорошо изучены и обеспечивают приемлемое качество в условиях умеренных помех. Однако их эффективность существенно снижается в сценариях с сильной интерференцией, быстрыми замираниями и в отсутствие точной информации о состоянии канала (Channel State Information, CSI). Метод наименьших квадратов (LS), будучи вычислительно простым, крайне чувствителен к шуму, в то время как более робастные методы, такие как LMMSE, требуют априорного знания статистики канала и шума, что зачастую недостижимо в реальных динамичных условиях.

В последние годы активно развиваются методы машинного и глубокого обучения, демонстрирующие значительный потенциал для оптимизации телекоммуникационных систем. В частности, нейросетевые приёмники, способные обучаться сквозному преобразованию принятого сигнала в битовые последовательности, показывают результаты, превосходящие классические методы. Одной из таких современных архитектур является нейросетевой приёмник от NVIDIA, построенный на основе остаточных свёрточных блоков (Residual Blocks), который позволяет эффективно компенсировать искажения сигнала без явного знания матрицы канала.

В данной работе предлагается модификация данного нейросетевого приёмника, направленная на дальнейшее повышение его эффективности. Модификация заключается во внедрении механизма внимания CBAM (Convolutional Block Attention Module) в структуру остаточных блоков. Этот механизм позволяет сети адаптивно выделять наиболее информативные признаки в частотно-временном и канальном пространствах, усиливая полезные компоненты сигнала и подавляя шумовые.

Экспериментальные исследования, проведённые на стандартизированных моделях каналов CDL (Clustered Delay Line), подтвердили эффективность предложенного подхода. Модифицированный приёмник продемонстрировал устойчивое улучшение показателя BLER (Block Error Rate) по сравнению как с классическими методами (LS, LMMSE), так и с базовой нейросетевой архитектурой, в особенности в условиях нелинейной видимости (NLOS) и при низких отношениях сигнал/шум. Таким образом, вклад работы заключается в разработке и валидации усовершенствованной архитектуры приёмника, что открывает пути для создания более помехоустойчивых и спектрально-эффективных систем связи следующего поколения.

\section*{2) Обзор литературы}

Современные системы связи пятого поколения требуют принципиально новых подходов к обработке сигналов на физическом уровне. Традиционные методы, такие как оценка канала и эквализация, хорошо изучены, но имеют существенные ограничения в условиях сложных сценариев распространения сигнала. \citet{badr2019lmmse} проводит всесторонний анализ LMMSE-оценки канала в контексте OFDM систем, подчеркивая её преимущества перед более простыми методами, такими как метод наименьших квадратов (LS), но также отмечая зависимость от точного знания статистических характеристик канала. Исследование \citet{bagadi2010mimo} демонстрирует применение пилот-сигналов для оценки канала в системах MIMO-OFDM, что является стандартным подходом в современных системах связи.

Архитектуры MIMO систем играют ключевую роль в повышении пропускной способности и надежности беспроводных каналов. \citet{shah2017performance} предоставляет сравнительный анализ различных конфигураций MIMO, включая SISO, SIMO, MISO и MIMO, показывая преимущества пространственного разнесения и мультиплексирования. Дальнейшее развитие MIMO технологий представлено в работе \citet{pessoa2020cdl}, где предлагается модель канала на основе CDL с двухполяризованными антеннами, специально адаптированная для сценариев 5G в сельской местности.

Теоретические основы достижимых скоростей передачи в системах связи продолжают развиваться. \citet{bjornson2015achievable} исследует предельные возможности Rician MIMO каналов с учетом аппаратных искажений, что особенно актуально для систем массового MIMO. В то же время \citet{caire2017achievable} фокусируется на методах вероятностного формирования сигналов (probabilistic shaping) для повышения спектральной эффективности.

Прорыв в области глубокого обучения открыл новые перспективы для оптимизации физического уровня. Архитектура глубоких остаточных свёрточных сетей, представленная в работе \citet{he2016deep}, стала фундаментом для многих последующих разработок, включая нейросетевые приемники. Механизмы внимания, такие как CBAM (Convolutional Block Attention Module) из исследования \citet{woo2018cbam}, демонстрируют эффективность в выделении информативных признаков в компьютерном зрении, что может быть успешно перенесено на задачи обработки сигналов.

Современные инструменты для исследований в области связи, такие как Sionna от NVIDIA \citep{sionna}, предоставляют мощную платформу для сквозного моделирования систем связи и интеграции методов машинного обучения. В частности, \citet{sionna2023neural} демонстрирует применение нейросетевых приемников для OFDM SIMO систем. Исследования \citet{levis2020deeprx} и \citet{levis2020end} развивают концепцию полностью сверточного глубокого приемника и сквозного обучения для OFDM систем, включая возможность беспилотной коммуникации.

Важным аспектом современных систем связи являются эффективные методы кодирования. Работы \citet{nguyen2019efficient} и \citet{nguyen2021low} посвящены разработке низкосложных декодеров QC-LDPC для 5G New Radio, что критически важно для практической реализации систем следующего поколения. Перспективным направлением является также совместное демодулирование сложных созвездий с использованием машинного обучения, как показано в исследовании \citet{gansekoele2025joint}.

Таким образом, современное состояние исследований демонстрирует четкий тренд к интеграции методов машинного обучения с классическими подходами обработки сигналов, что открывает новые возможности для создания более эффективных и устойчивых систем связи.

\section*{3) Математическая постановка задачи}
Рассматривается система MIMO с \( N_\text{t} \) передающими антеннами и \( N_\text{r} \) принимающими антеннами. Пусть:
\begin{equation}
\mathbf{x} = \mathbf{W} \mathbf{s}
\end{equation}
где \(\mathbf{W} \in \mathbb{C}^{N_\text{t} \times N_\text{r}}\) – матрица прекодера, \(\mathbf{s} \in \mathbb{C}^{N_\text{r} \times 1}\) – вектор исходных сигналов ( бинарная последовательность ), а \(\mathbf{x} \in \mathbb{C}^{N_\text{t} \times 1}\) - передаваемый вектор сигналов после прекодера (каждый символ в пропорции распространён по всем антеннам). Этот вектор обычно берётся из некоторой M-QAM/QPSK сигнальной диаграммы, т.\,е. \( \mathbf{x} \) состоит из комплексных значений, отражающих точку созвездия.

Модель прохождения сигнала через канал можно записать в следующем виде:

\begin{equation}
\mathbf{y} = \mathbf{H} \mathbf{x} + \mathbf{n},
\end{equation}
где
\[
\mathbf{y} \in \mathbb{C}^{N_\text{r} \times 1}
\]
– вектор принятых сигналов на антенных элементах приёмника, \(\mathbf{H} \in \mathbb{C}^{N_\text{r} \times N_\text{t}}\) – матрица канала, описывающая затухание и фазовые сдвиги на пути «от каждой передающей антенны к каждой приёмной», а \(\mathbf{n} \in \mathbb{C}^{N_\text{r} \times 1}\) – вектор аддитивного белого гауссовского шума (AWGN) с дисперсией \(\sigma_n^2\).

В классической постановке задачи \textbf{MIMO-декодирования} (или «символьной детекции») требуется найти оценки \(\hat{\mathbf{x}}\) переданных символов, обладающие минимальной вероятностью ошибки. При стандартном подходе (например, методе MMSE – Minimum Mean Square Error) оценивается:
\begin{equation}
\hat{\mathbf{x}}_\mathrm{MMSE} = \underset{\mathbf{x}}{\mathrm{argmin}}( 
\Big\lVert \mathbf{y} - \mathbf{H}\mathbf{x} \Big\rVert^2 + \sigma_n^2 \lVert \mathbf{x} \rVert^2).
\end{equation}
Для высоких размерностей (большое число антенн \(N_\text{t}\) и \(N_\text{r}\)), а также для сложных созвездий (например, 64-QAM, 256-QAM) точный поиск \(\mathbf{x}\) может становиться вычислительно дорогостоящим. Методы глубокого обучения позволяют решать данную задачу итеративно или напрямую, обучаясь на большом наборе «принятый сигнал – истинный символ» без явной оптимизации по формуле минимума ошибки.

В рамках предлагаемой работы рассматривается ситуация, когда матрица канала \(\mathbf{H}\) частично неизвестна или может изменяться со временем, а число передатчиков и приёмников велико. Требуется реализовать «глубокую» нейронную сеть, которая сможет, во-первых, оценивать параметры \(\mathbf{H}\) по пилотным символам, а во-вторых, автоматически восстанавливать \(\mathbf{x}\) из \(\mathbf{y}\) с учётом сложности канала и помех.

\textbf{Основная цель} формализуется как задача:
\begin{equation}
\hat{\mathbf{x}} = \mathcal{F}_\theta(\mathbf{y}, \mathbf{p}),
\end{equation}
где \(\mathbf{p}\) – набор пилотных символов, а \(\mathcal{F}_\theta\) – функция, задаваемая параметрами нейронной сети \(\theta\). 

Требуется найти такие параметры \(\theta\), которые минимизируют среднюю вероятность ошибки детектирования:
\begin{equation}
\theta^* = \underset{\theta}{\mathrm{argmin}} \; \mathbb{E}_{(\mathbf{x},\mathbf{H},\mathbf{n})} 
\Big\{ \ell\big(\mathcal{F}_\theta(\mathbf{y}, \mathbf{p}),\, \mathbf{x}\big) \Big\},
\end{equation}
где \(\ell(\cdot)\) – функция потерь (кросс-энтропия).

Для оценки качества декодирования в моделях связи используется BLER (Block Error Rate): 

\begin{equation}
\text{BLER} = \frac{\text{Количество ошибочных блоков}}{\text{Общее количество блоков}} 
\end{equation}

Ошибочный блок — это блок, в котором хотя бы один бит не совпадает с исходным. BLER помогает отслеживать эффективность модели, показывая, насколько часто возникают ошибки в передаваемых данных. Чем ниже BLER, тем лучше модель справляется с исправлением ошибок.

\clearpage
\bibliographystyle{plainnat}
\bibliography{references}


\end{document}